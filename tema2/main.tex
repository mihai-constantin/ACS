\documentclass[titlepage, a4paper,12pt]{article}

% cod folosit din arhiva: line 7 -> line 77

\usepackage{tikz}
\usepackage[siunitx]{circuitikz}
\usetikzlibrary{calc}    
\tikzset{
  dim above/.style={to path={\pgfextra{
        \pgfinterruptpath
        \draw[>=latex,|<->|] let
        \p1=($(\tikztostart)!2mm!90:(\tikztotarget)$),
        \p2=($(\tikztotarget)!2mm!-90:(\tikztostart)$)
        in(\p1) -- (\p2) node[pos=.5,sloped,above]{#1};
        \endpgfinterruptpath
      }(\tikztostart) -- (\tikztotarget) \tikztonodes
    }
  },
  dim below/.style={to path={\pgfextra{
        \pgfinterruptpath
        \draw[>=latex,|<->|] let 
        \p1=($(\tikztostart)!2mm!90:(\tikztotarget)$),
        \p2=($(\tikztotarget)!2mm!-90:(\tikztostart)$)
        in (\p1) -- (\p2) node[pos=.5,sloped,below]{#1};
        \endpgfinterruptpath
      }(\tikztostart) -- (\tikztotarget) \tikztonodes
    }
  },
}

\usepackage{graphicx}
\graphicspath{ {images/} } % director pentru imaginile pe care le inserez in document
\usepackage[romanian]{babel}
\usepackage[unicode]{hyperref} 
\usepackage{amsmath,epsfig,pifont,calc,pifont,pstricks}
\usepackage{rom} 						 % pentru a scrie cu diacritice in limba romana	
\usepackage{color}
\usepackage{listings}
\definecolor{mygreen}{RGB}{28,172,0} % color values Red, Green, Blue
\definecolor{mylilas}{RGB}{170,55,241}
\lstset{language=Matlab,%
    %basicstyle=\color{red},
    breaklines=true,%
    morekeywords={matlab2tikz},
    keywordstyle=\color{blue},%
    morekeywords=[2]{1}, keywordstyle=[2]{\color{black}},
    identifierstyle=\color{black},%
    stringstyle=\color{mylilas},
    commentstyle=\color{mygreen},%
    showstringspaces=false,%without this there will be a symbol in the places where there is a space
    numbers=left,%
    numberstyle={\tiny \color{black}},% size of the numbers
    numbersep=9pt, % this defines how far the numbers are from the text
    emph=[1]{for,end,break},emphstyle=[1]\color{red}, %some words to emphasise
    %emph=[2]{word1,word2}, emphstyle=[2]{style},    
}

%%% setari ale paginii
%indentarea la inceput de paragraf
\setlength{\parindent}{3ex}

%dimensiunea textului pe pagina
\setlength{\voffset}{-2cm}
\setlength{\textheight}{23cm}  
\setlength{\textwidth}{16cm}
\setlength{\topmargin}{1cm}
\setlength{\headsep}{1cm}
\setlength{\headheight}{15pt}
\renewcommand{\baselinestretch}{1.2}
\newcommand{\myindent}{\hspace*{3ex}}
%\renewcommand{\baselinestretch}{1}

%margini
\setlength{\oddsidemargin}{0.5cm}
\setlength{\evensidemargin}{-0.3cm}
%\raggedright
\raggedbottom

\usepackage{extarrows}
\usepackage{multicol} % scriere pe mai multe coloane
\usepackage{siunitx} % sist international de mas
 
\usepackage[nottoc]{tocbibind}

\newcommand\spatiu[1][0.2cm]{\hspace*{#1}} % tab
\newcommand\tab[1][0.7cm]{\hspace*{#1}} % tab
\newcommand\TAB[1][1.2cm]{\hspace*{#1}} % TAB

\renewcommand\tablename{Tabelul}  % altfel scrie "Tabela"
%\renewcommand\bibname{Referin'te}  % altfel scrie "Bibliografie" si acum referintele includ si pagini web

\usepackage{fancyhdr}
\pagestyle{fancy}
\fancyhf{}
\lhead{Tema 2 - BE}
\rhead{Constantin Mihai - 311 CD}

\rfoot{\centering\thepage} % setare nr pagina centru

\title{\line(1,0){350}\\ \Huge Bazele electrotehnicii \\ 
\huge Tema 2 \\
\line(1,0){350}
}
\author{{\em \Large Constantin Mihai} \\
\large 311 CD \\ \\
{\large Facultatea de Automatic\u{a} \c{s}i Calculatoare} \\
\large \texttt{mihai.constantin98@gmail.com} 
}
\date{\today}

\begin{document}

\hbadness=1000000

\maketitle

%%%%%%%%%%%%%%%%%%%%%%%%%%%%%%%%%%%%%%%%%%%%%%%%%%%%%%%%%%%%%%%%%%%%%%%%%%%%%%%%%%%%%%%%%%%%%
%%%%%%%%%%%%%%%%%%                Realizator: Pop Adrian              %%%%%%%%%%%%%%%%%%%%%%%
%%%%%%%%%%%%%%%%%%%%%%%%%%%%%%%%%%%%%%%%%%%%%%%%%%%%%%%%%%%%%%%%%%%%%%%%%%%%%%%%%%%%%%%%%%%%%
\usetikzlibrary{backgrounds}
\makeatletter

%% \pgf@circ@Rlen = \pgfkeysvalueof{/tikz/circuitikz/bipoles/length}
%% Commenting the line above apparently solved the missing "Missing number, treated as zero" error

\def\TikzBipolePath#1#2{\pgf@circ@bipole@path{#1}{#2}}
\makeatother

\newlength{\ResUp} 
\newlength{\ResRight}

%%%%%%%%%%%%%%%%%%%%%%%%%%%%%%%%%%%%%%%%%%%%%%%%%%%%%%%%%%%%%%%%%%%%%%%%%%%%%%%%%%%%%%%%%%%%%
%%%%%%%%%%%%%%%%%%      Sursa de curent: simbol folosit in Romania    %%%%%%%%%%%%%%%%%%%%%%%
%%%%%%%%%%%%%%%%%%%%%%%%%%%%%%%%%%%%%%%%%%%%%%%%%%%%%%%%%%%%%%%%%%%%%%%%%%%%%%%%%%%%%%%%%%%%%
%Declaram dimensiunile initiale
\ctikzset{bipoles/romanianCurrentSource/height/.initial=.60}
\ctikzset{bipoles/romanianCurrentSource/width/.initial=.60}

%Definim noul simbol pentru SIC
\pgfcircdeclarebipole{} 
	%Offset pentru label-uri
	{\ctikzvalof{bipoles/romanianCurrentSource/height}}
	%Numele simbolului
	{romanianCurrentSource}
	%Dimensiunile "cutiei" in care va sta
	{\ctikzvalof{bipoles/romanianCurrentSource/height}}
	{\ctikzvalof{bipoles/romanianCurrentSource/width}}
	{
		%Stabilim grosimea standard a liniei pentru un element
		\pgfsetlinewidth{\pgfkeysvalueof{/tikz/circuitikz/bipoles/thickness}\pgfstartlinewidth}
		
		%Definim ancorele
		\pgfextracty{\ResUp}{\northeast}
		\pgfextractx{\ResRight}{\southwest}
		
		%Desenam cerculetul
		\pgfpathellipse{\pgfpointorigin}{\pgfpoint{0cm}{\ResUp}}{\pgfpoint{\ResRight}{0cm}}
		
		%Desenam prima sagetica; trebuie sa avem grija ca dupa ce desenam un segment,
		%urmatoarea linie va fi desenata relativ la capatul segmentului, nu fata de
		%origine, deci sunt necesare cateva repozitionari
		\pgfmoveto{\pgfpoint{1.0\ResRight}{0.0\ResUp}}   %Ne pozitionam in (-1, 0)
		\pgflineto{\pgfpoint{0.1\ResRight}{0.0\ResUp}}   %Desenam corpul sagetii
		\pgflineto{\pgfpoint{0.3\ResRight}{-0.25\ResUp}} %Desenam unul din capete
		\pgfmoveto{\pgfpoint{0.1\ResRight}{0.0\ResUp}}   %Ne pozitionam in (-0.1, 0)
	    \pgflineto{\pgfpoint{0.3\ResRight}{0.25\ResUp}}  %Desenam celalalt capat
		
		%Desenam ce-a de-a doua sagetica
		\pgfmoveto{\pgfpoint{-0.2\ResRight}{0.0\ResUp}}  %Ne pozitionam in (0.2, 0)
		\pgflineto{\pgfpoint{-1.0\ResRight}{0.0\ResUp}}  %Desenam corpul sagetii
		\pgfmoveto{\pgfpoint{0.0\ResRight}{0.25\ResUp}}  %Ne repozitionam
		\pgflineto{\pgfpoint{-0.2\ResRight}{0.0\ResUp}}  %Desenam unul din capete
		\pgflineto{\pgfpoint{0.0\ResRight}{-0.25\ResUp}} %Desenam celalat capat
		
		%Pentru desenare, sa folosim functia draw
		\pgfusepath{draw}
	}

%Ii definim un stil si o cale si...gata!
\def\romanianCurrentSourcepath#1{\TikzBipolePath{romanianCurrentSource}{#1}}
\tikzset{romanianCurrentSource/.style = {\circuitikzbasekey, /tikz/to path=\romanianCurrentSourcepath, l=#1}}


%%%%%%%%%%%%%%%%%%%%%%%%%%%%%%%%%%%%%%%%%%%%%%%%%%%%%%%%%%%%%%%%%%%%%%%%%%%%%%%%%%%%%%%%%%%%%
%%%%%%%%%%%%%%%%%%      Sursa de tensiune: simbol folosit in Romania    %%%%%%%%%%%%%%%%%%%%
%%%%%%%%%%%%%%%%%%%%%%%%%%%%%%%%%%%%%%%%%%%%%%%%%%%%%%%%%%%%%%%%%%%%%%%%%%%%%%%%%%%%%%%%%%%%%
%Declaram dimensiunile initiale
\ctikzset{bipoles/romanianVoltageSource/height/.initial=.60}
\ctikzset{bipoles/romanianVoltageSource/width/.initial=.60}

%Definim noul simbol pentru SIT
\pgfcircdeclarebipole{}
	%Stabilim grosimea standard a liniei pentru un element
	{\ctikzvalof{bipoles/romanianVoltageSource/height}}
	%Numele simbolului
	{romanianVoltageSource}
	%Dimensiunile "cutiei" in care va sta
	{\ctikzvalof{bipoles/romanianVoltageSource/height}}
	{\ctikzvalof{bipoles/romanianVoltageSource/width}}
	{
		%Stabilim grosimea standard a liniei pentru un element
		\pgfsetlinewidth{\pgfkeysvalueof{/tikz/circuitikz/bipoles/thickness}\pgfstartlinewidth}
		
		%Definim ancorele
		\pgfextracty{\ResUp}{\northeast}
		\pgfextractx{\ResRight}{\southwest}
		
		%Desenam cerculetul
		\pgfpathellipse{\pgfpointorigin}{\pgfpoint{0}{\ResUp}}{\pgfpoint{\ResRight}{0}}
		
		%Desenam prima sagetica; trebuie sa avem grija ca dupa ce desenam un segment,
		%urmatoarea linie va fi desenata relativ la capatul segmentului, nu fata de
		%origine, deci sunt necesare cateva repozitionari
		\pgfmoveto{\pgfpoint{1.0\ResRight}{0.0\ResUp}}   %Ne pozitionam in (-1, 0)
		\pgflineto{\pgfpoint{-1.0\ResRight}{0.0\ResUp}}   %Desenam corpul sagetii
		\pgflineto{\pgfpoint{-0.7\ResRight}{-0.25\ResUp}} %Desenam unul din capete
		\pgfmoveto{\pgfpoint{-1.0\ResRight}{0.0\ResUp}}   %Ne pozitionam in (-0.1, 0)
	    \pgflineto{\pgfpoint{-0.7\ResRight}{0.25\ResUp}}  %Desenam celalalt capat
		
		%Pentru desenare, sa folosim functia draw
		\pgfusepath{draw}
	}
	\def\romanianVoltageSourcepath#1{\TikzBipolePath{romanianVoltageSource}{#1}}
%Ii stabilim un nume, un posibil label si...gata!
\tikzset{romanianVoltageSource/.style = {\circuitikzbasekey, /tikz/to path=\romanianVoltageSourcepath, l=#1}}


%%%%%%%%%%%%%%%%%%%%%%%%%%%%%%%%%%%%%%%%%%%%%%%%%%%%%%%%%%%%%%%%%%%%%%%%%%%%%%%%%%%%%%%%%%%%%
%%%%%%%%%%%%%%      Sursa comandata de curent: simbol folosit in Romania    %%%%%%%%%%%%%%%%%
%%%%%%%%%%%%%%%%%%%%%%%%%%%%%%%%%%%%%%%%%%%%%%%%%%%%%%%%%%%%%%%%%%%%%%%%%%%%%%%%%%%%%%%%%%%%%
%Declaram dimensiunile initiale
\ctikzset{bipoles/romanianCCS/height/.initial=.60}
\ctikzset{bipoles/romanianCCS/width/.initial=.60}

%Definim noul simbol pentru SCC
\pgfcircdeclarebipole{} 
	%Offset pentru label-uri
	{\ctikzvalof{bipoles/romanianCCS/height}}
	%Numele simbolului
	{romanianCCS}
	%Dimensiunile "cutiei" in care va sta
	{\ctikzvalof{bipoles/romanianCCS/height}}
	{\ctikzvalof{bipoles/romanianCCS/width}}
	{
		%Stabilim grosimea standard a liniei pentru un element
		\pgfsetlinewidth{\pgfkeysvalueof{/tikz/circuitikz/bipoles/thickness}\pgfstartlinewidth}
		
		%Definim ancorele
		\pgfextracty{\ResUp}{\northeast}
		\pgfextractx{\ResRight}{\southwest}
		
		%Desenam rombul
		\pgftransformrotate{-45}
		\pgfpathrectanglecorners{\southwest}{\northeast}
		\pgftransformrotate{45}

		%Desenam prima sagetica; trebuie sa avem grija ca dupa ce desenam un segment,
		%urmatoarea linie va fi desenata relativ la capatul segmentului, nu fata de
		%origine, deci sunt necesare cateva repozitionari
		\pgfmoveto{\pgfpoint{1.25\ResRight}{0.0\ResUp}}   %Ne pozitionam in (-1, 0)
		\pgflineto{\pgfpoint{0.1\ResRight}{0.0\ResUp}}   %Desenam corpul sagetii
		\pgflineto{\pgfpoint{0.3\ResRight}{-0.25\ResUp}} %Desenam unul din capete
		\pgfmoveto{\pgfpoint{0.1\ResRight}{0.0\ResUp}}   %Ne pozitionam in (-0.1, 0)
	    \pgflineto{\pgfpoint{0.3\ResRight}{0.25\ResUp}}  %Desenam celalalt capat
		
		%Desenam ce-a de-a doua sagetica
		\pgfmoveto{\pgfpoint{-0.2\ResRight}{0.0\ResUp}}  %Ne pozitionam in (0.2, 0)
		\pgflineto{\pgfpoint{-1.25\ResRight}{0.0\ResUp}}  %Desenam corpul sagetii
		\pgfmoveto{\pgfpoint{0.0\ResRight}{0.25\ResUp}}  %Ne repozitionam
		\pgflineto{\pgfpoint{-0.2\ResRight}{0.0\ResUp}}  %Desenam unul din capete
		\pgflineto{\pgfpoint{0.0\ResRight}{-0.25\ResUp}} %Desenam celalat capat
		
		%Pentru desenare, sa folosim functia draw
		\pgfusepath{draw}
	}

%Ii definim un stil si o cale si...gata!
\def\romanianCCS#1{\TikzBipolePath{romanianCCS}{#1}}
\tikzset{romanianCCS/.style = {\circuitikzbasekey, /tikz/to path=\romanianCCS, l=#1}}


%%%%%%%%%%%%%%%%%%%%%%%%%%%%%%%%%%%%%%%%%%%%%%%%%%%%%%%%%%%%%%%%%%%%%%%%%%%%%%%%%%%%%%%%%%%%%
%%%%%%%%%%%%%      Sursa comandata de tensiune: simbol folosit in Romania    %%%%%%%%%%%%%%%%
%%%%%%%%%%%%%%%%%%%%%%%%%%%%%%%%%%%%%%%%%%%%%%%%%%%%%%%%%%%%%%%%%%%%%%%%%%%%%%%%%%%%%%%%%%%%%
%Declaram dimensiunile initiale
\ctikzset{bipoles/romanianCVS/height/.initial=.60}
\ctikzset{bipoles/romanianCVS/width/.initial=.60}

%Definim noul simbol pentru CVS
\pgfcircdeclarebipole{}
	%Stabilim grosimea standard a liniei pentru un element
	{\ctikzvalof{bipoles/romanianCVS/height}}
	%Numele simbolului
	{romanianCVS}
	%Dimensiunile "cutiei" in care va sta
	{\ctikzvalof{bipoles/romanianCVS/height}}
	{\ctikzvalof{bipoles/romanianCVS/width}}
	{
		%Stabilim grosimea standard a liniei pentru un element
		\pgfsetlinewidth{\pgfkeysvalueof{/tikz/circuitikz/bipoles/thickness}\pgfstartlinewidth}
		
		%Definim ancorele
		\pgfextracty{\ResUp}{\northeast}
		\pgfextractx{\ResRight}{\southwest}
		
		%Desenam rombul
		\pgftransformrotate{-45}
		\pgfpathrectanglecorners{\southwest}{\northeast}
		\pgftransformrotate{45}
		
		%Desenam prima sagetica; trebuie sa avem grija ca dupa ce desenam un segment,
		%urmatoarea linie va fi desenata relativ la capatul segmentului, nu fata de
		%origine, deci sunt necesare cateva repozitionari
		\pgfmoveto{\pgfpoint{1.25\ResRight}{0.0\ResUp}}   %Ne pozitionam in (-1.25, 0)
		\pgflineto{\pgfpoint{-1.3\ResRight}{0.0\ResUp}}  %Desenam corpul sagetii
		\pgflineto{\pgfpoint{-0.9\ResRight}{-0.25\ResUp}} %Desenam unul din capete
		\pgfmoveto{\pgfpoint{-1.3\ResRight}{0.0\ResUp}}   %Ne pozitionam in (-0.1, 0)
	    \pgflineto{\pgfpoint{-0.9\ResRight}{0.25\ResUp}}  %Desenam celalalt capat
		
		%Pentru desenare, sa folosim functia draw
		\pgfusepath{draw}
	}
	\def\romanianCVS#1{\TikzBipolePath{romanianCVS}{#1}}
%Ii stabilim un nume, un posibil label si...gata!
\tikzset{romanianCVS/.style = {\circuitikzbasekey, /tikz/to path=\romanianCVS, l=#1}}


%%%%%%%%%%%%%%%%%%%%%%%%%%%%%%%%%%%%%%%%%%%%%%%%%%%%%%%%%%%%%%%%%%%%%%%%%%%%%%%%%%%%%%%%%%%%%
%%%%%%%%%%%%%%%%%      Dioda Zenner 1 & 2: simbol folosit in Romania    %%%%%%%%%%%%%%%%%%%%%
%%%%%%%%%%%%%%%%%%%%%%%%%%%%%%%%%%%%%%%%%%%%%%%%%%%%%%%%%%%%%%%%%%%%%%%%%%%%%%%%%%%%%%%%%%%%%
%Declaram dimensiunile initiale
\ctikzset{bipoles/zDoZ/height/.initial=.60}
\ctikzset{bipoles/zDoZ/width/.initial=1.00}
  
%Definim noul simbol pentru diodaZenner
\pgfcircdeclarebipole{}
	%Stabilim grosimea standard a liniei pentru un element
	{\ctikzvalof{bipoles/zDoZ/height}}
	%Numele simbolului
	{zDoZ}
	%Dimensiunile "cutiei" in care va sta
	{\ctikzvalof{bipoles/zDoZ/height}}
	{\ctikzvalof{bipoles/zDoZ/width}}
	{
		%Stabilim grosimea standard a liniei pentru un element
		\pgfsetlinewidth{\pgfkeysvalueof{/tikz/circuitikz/bipoles/thickness}\pgfstartlinewidth}
		
		%Definim ancorele
		\pgfextracty{\ResUp}{\northeast}
		\pgfextractx{\ResRight}{\southwest}
		
		%Desenam, fara alte explicatii, ca acum stim
		\pgfmoveto{\pgfpoint{0.0\ResRight}{0.0\ResUp}}
		\pgflineto{\pgfpoint{1.0\ResRight}{0.6\ResUp}}
		\pgflineto{\pgfpoint{1.0\ResRight}{-0.6\ResUp}}
		\pgflineto{\pgfpoint{-1.0\ResRight}{0.6\ResUp}}
		\pgflineto{\pgfpoint{-1.0\ResRight}{-0.6\ResUp}}
		
		%Pentru desenare, sa folosim functia draw
		\pgfusepath{draw}
	}
	\def\zDoZ#1{\TikzBipolePath{zDoZ}{#1}}
%Ii stabilim un nume, un posibil label si...gata!
\tikzset{zDoZ/.style = {\circuitikzbasekey, /tikz/to path=\zDoZ, l=#1}}


%Declaram dimensiunile initiale
\ctikzset{bipoles/zDoZZ/height/.initial=.60}
\ctikzset{bipoles/zDoZZ/width/.initial=1.00}
  
%Definim noul simbol pentru diodaZennerInversa
\pgfcircdeclarebipole{}
	%Stabilim grosimea standard a liniei pentru un element
	{\ctikzvalof{bipoles/zDoZZ/height}}
	%Numele simbolului
	{zDoZZ}
	%Dimensiunile "cutiei" in care va sta
	{\ctikzvalof{bipoles/zDoZZ/height}}
	{\ctikzvalof{bipoles/zDoZZ/width}}
	{
		%Stabilim grosimea standard a liniei pentru un element
		\pgfsetlinewidth{\pgfkeysvalueof{/tikz/circuitikz/bipoles/thickness}\pgfstartlinewidth}
		
		%Definim ancorele
		\pgfextracty{\ResUp}{\northeast}
		\pgfextractx{\ResRight}{\southwest}
		
		%Desenam, fara alte explicatii, ca acum stim
		\pgfmoveto{\pgfpoint{0.0\ResRight}{0.0\ResUp}}
		\pgflineto{\pgfpoint{1.0\ResRight}{-0.6\ResUp}}
		\pgflineto{\pgfpoint{1.0\ResRight}{0.6\ResUp}}
		\pgflineto{\pgfpoint{-1.0\ResRight}{-0.6\ResUp}}
		\pgflineto{\pgfpoint{-1.0\ResRight}{0.6\ResUp}}
		
		%Pentru desenare, sa folosim functia draw
		\pgfusepath{draw}
	}
	\def\zDoZZ#1{\TikzBipolePath{zDoZZ}{#1}}
%Ii stabilim un nume, un posibil label si...gata!
\tikzset{zDoZZ/.style = {\circuitikzbasekey, /tikz/to path=\zDoZZ, l=#1}}


%%%%%%%%%%%%%%%%%%%%%%%%%%%%%%%%%%%%%%%%%%%%%%%%%%%%%%%%%%%%%%%%%%%%%%%%%%%%%%%%%%%%%%%%%%%%%
%%%%%%%%%%%%%%%%      Condensator neliniar: simbol folosit in Romania    %%%%%%%%%%%%%%%%%%%%
%%%%%%%%%%%%%%%%%%%%%%%%%%%%%%%%%%%%%%%%%%%%%%%%%%%%%%%%%%%%%%%%%%%%%%%%%%%%%%%%%%%%%%%%%%%%%
%Declaram dimensiunile initiale
\ctikzset{bipoles/Cn/height/.initial=.40}
\ctikzset{bipoles/Cn/width/.initial=1.00}

%Definim noul simbol pentru condNeliniar
\pgfcircdeclarebipole{}
	%Stabilim grosimea standard a liniei pentru un element
	{\ctikzvalof{bipoles/Cn/height}}
	%Numele simbolului
	{Cn}
	%Dimensiunile "cutiei" in care va sta
	{\ctikzvalof{bipoles/Cn/height}}
	{\ctikzvalof{bipoles/Cn/width}}
	{
		%Stabilim grosimea standard a liniei pentru un element
		\pgfsetlinewidth{\pgfkeysvalueof{/tikz/circuitikz/bipoles/thickness}\pgfstartlinewidth}
		
		%Definim ancorele
		\pgfextracty{\ResUp}{\northeast}
		\pgfextractx{\ResRight}{\southwest}
		
		%Desenam dreptunghiul
		\pgfpathrectanglecorners{\southwest}{\northeast}
		
		%Desenam cele doua linii paralele
		\pgfmoveto{\pgfpoint{0.2\ResRight}{0.8\ResUp}}
		\pgflineto{\pgfpoint{0.2\ResRight}{-0.8\ResUp}}
		
		\pgfmoveto{\pgfpoint{-0.2\ResRight}{0.8\ResUp}}
		\pgflineto{\pgfpoint{-0.2\ResRight}{-0.8\ResUp}}
		
		%Desenam liniile de pe mijloc
		\pgfmoveto{\pgfpoint{1.0\ResRight}{0.0\ResUp}}
		\pgflineto{\pgfpoint{0.2\ResRight}{0.0\ResUp}}
		
		\pgfmoveto{\pgfpoint{-0.2\ResRight}{0.0\ResUp}}
		\pgflineto{\pgfpoint{-1.0\ResRight}{0.0\ResUp}}
		
		%Pentru desenare, sa folosim functia draw
		\pgfusepath{draw}
		
		%Desenam chestiuta neagra
		\pgfsetlinewidth{5.0\pgfkeysvalueof{/tikz/circuitikz/bipoles/thickness}\pgfstartlinewidth}
		\pgfmoveto{\pgfpoint{-1.0\ResRight}{1.0\ResUp}}
		\pgflineto{\pgfpoint{-1.0\ResRight}{-1.0\ResUp}}
		\pgfmoveto{\pgfpoint{-0.9\ResRight}{1.0\ResUp}}
		\pgflineto{\pgfpoint{-0.9\ResRight}{-1.0\ResUp}}
		\pgfmoveto{\pgfpoint{-0.8\ResRight}{1.0\ResUp}}
		\pgflineto{\pgfpoint{-0.8\ResRight}{-1.0\ResUp}}
		
		%Pentru desenare, sa folosim functia draw
		\pgfusepath{draw}
	}
	\def\Cn#1{\TikzBipolePath{Cn}{#1}}
%Ii stabilim un nume, un posibil label si...gata!
\tikzset{Cn/.style = {\circuitikzbasekey, /tikz/to path=\Cn, l=#1}}


%%%%%%%%%%%%%%%%%%%%%%%%%%%%%%%%%%%%%%%%%%%%%%%%%%%%%%%%%%%%%%%%%%%%%%%%%%%%%%%%%%%%%%%%%%%%%
%%%%%%%%%%%%%%%%%%      Bobina neliniara: simbol folosit in Romania    %%%%%%%%%%%%%%%%%%%%%%
%%%%%%%%%%%%%%%%%%%%%%%%%%%%%%%%%%%%%%%%%%%%%%%%%%%%%%%%%%%%%%%%%%%%%%%%%%%%%%%%%%%%%%%%%%%%%
%Declaram dimensiunile initiale
\ctikzset{bipoles/Ln/height/.initial=.40}
\ctikzset{bipoles/Ln/width/.initial=1.00}

%Definim noul simbol pentru condNeliniar
\pgfcircdeclarebipole{}
	%Stabilim grosimea standard a liniei pentru un element
	{\ctikzvalof{bipoles/Ln/height}}
	%Numele simbolului
	{Ln}
	%Dimensiunile "cutiei" in care va sta
	{\ctikzvalof{bipoles/Ln/height}}
	{\ctikzvalof{bipoles/Ln/width}}
	{
		%Stabilim grosimea standard a liniei pentru un element
		\pgfsetlinewidth{\pgfkeysvalueof{/tikz/circuitikz/bipoles/thickness}\pgfstartlinewidth}
		
		%Definim ancorele
		\pgfextracty{\ResUp}{\northeast}
		\pgfextractx{\ResRight}{\southwest}
		
		%Desenam dreptunghiul
		\pgfpathrectanglecorners{\southwest}{\northeast}
	
		%Desenam spirele
		\pgfmoveto{\pgfpoint{1.0\ResRight}{0.0\ResUp}}
		\pgflineto{\pgfpoint{0.80\ResRight}{0.0\ResUp}}
		\pgfpatharc{0}{-180}{\ResRight/6}
		\pgfpatharc{0}{-180}{\ResRight/6}
		\pgfpatharc{0}{-180}{\ResRight/6}
		\pgfpatharc{0}{-180}{\ResRight/6}
		\pgflineto{\pgfpoint{-0.8\ResRight}{0.0\ResUp}}
		
		%Pentru desenare, sa folosim functia draw
		\pgfusepath{draw}
		
		%Desenam chestiuta neagra
		\pgfsetlinewidth{5.3\pgfkeysvalueof{/tikz/circuitikz/bipoles/thickness}\pgfstartlinewidth}
		\pgfmoveto{\pgfpoint{-1.0\ResRight}{1.0\ResUp}}
		\pgflineto{\pgfpoint{-1.0\ResRight}{-1.0\ResUp}}
		\pgfmoveto{\pgfpoint{-0.9\ResRight}{1.0\ResUp}}
		\pgflineto{\pgfpoint{-0.9\ResRight}{-1.0\ResUp}}
		\pgfmoveto{\pgfpoint{-0.8\ResRight}{1.0\ResUp}}
		\pgflineto{\pgfpoint{-0.8\ResRight}{-1.0\ResUp}}

		
		%Pentru desenare, sa folosim functia draw
		\pgfusepath{draw}
	}
	\def\Ln#1{\TikzBipolePath{Ln}{#1}}
%Ii stabilim un nume, un posibil label si...gata!
\tikzset{Ln/.style = {\circuitikzbasekey, /tikz/to path=\Ln, l=#1}}




\tableofcontents

\newpage

\section*{Enun\c{t}}

\paragraph{1. Rezolvarea circuitelor de c.a.} \mbox{} 

'In problema de circuit (f'ar'a surse comandate) pe care a'ti inventat-o la tema 1

a) \begin{itemize}
  \item Ad'auga'ti 'in serie cu un rezistor $R_1>0$ o bobin'a cu inductivitatea $L=x*100/\pi\spatiu\mathrm{mH}$, unde $x$ este egal'a numeric cu $R_1$.
  \item Ad'auga'ti 'in paralel cu un rezistor $R_2>0$ un condensator cu capacitatea $C=y*100/\pi \spatiu\mu\mathrm{F}$, unde $y$ este egal'a numeric cu $R_2$.
  \item Considera'ti $f=50\spatiu\mathrm{Hz}$.
  \item Schimba'ti toate sursele independente 'in surse sinusoidale, cu frecven't'a $f$ 'si expresia $x(t)=X\sqrt{2}sin(\omega t+\varphi)$, unde $X$ este valoarea pe care o avusese sursa 'in c.c., iar pentru $\varphi$ alege'ti valori diferite din mul'timea $\big\{0,\pm\pi/2,\pm\pi/4\big\}$.
  \item Desena'ti reprezentarea 'in complex a circuitului.
  \item Aplica'ti aceea'si metod'a de rezolvare pe care a'ti folosit-o la tema 1 'si scrie'ti sistemul de ecua'tii de rezolvat 'in complex.
  \item Scrie'ti un mic program Matlab/Octave pentru rezolvarea acestei probleme.
\end{itemize}

b) Completa'ti codul cu instruc'tiuni pentru verificarea bilan'tului de puteri 'in complex. \\ \par

c) Ilustra'ti trecerea 'in timp a m'arimilor complexe pentru una din m'arimile ob'tinute.

\paragraph{2. Rezolvarea circuitelor 'in regim tranzitoriu} \mbox{}

'In problema de circuit (f'ar'a surse comandate) pe care a'ti inventat-o la tema 1

a) \begin{itemize}
  \item Ad'auga'ti 'in serie cu un rezistor $R_1>0$ o bobin'a cu inductivitatea $L=x*100/\pi\spatiu\mathrm{mH}$, unde $x$ este egal'a numeric cu $R_1$ (pute'ti p'astra aceea'si bobin'a ca la punctul 1).
  \item Ad'auga'ti 'in paralel cu un rezistor $R_2>0$ un condensator cu capacitatea $C=y*100/\pi \spatiu\mu\mathrm{F}$, unde $y$ este egal'a numeric cu $R_2$ (pute'ti p'astra acela'si condensator ca la punctul 1).
  \item Presupune'ti c'a la momentul $t=0$ apare un "defect" 'si se rupe o latur'a din circuit, alta dec\^at cea pe care a'ti 'inseriat bobina sau cea cu care a'ti conectat 'in paralel condensatorul.
  \item Desena'ti schema 'in opera'tional a circuitului.
  \item Scrie'ti sistemul de rezolvat 'si rezolva'ti-l simbolic 'intr-un mediu potrivit (Matlab, Octave).
\end{itemize}

b) Analiza'ti starea final'a 'si verifica'ti teoremele valorilor ini'tiale 'si finale. \\ \par

c) Calcula'ti expresia instantanee a uneia dintre cele dou'a variabile de stare 'si reprezenta'ti-o grafic.

\paragraph{3. Calculul 'si reprezentarea unui c\^amp electric} \mbox{}

\begin{itemize}
  \item Alege'ti o distribu'tie de sarcin'a care s'a depind'a numai de raz'a 'intr-un sistem de coordonate sferic.
    \[   
\rho(r, \theta, \phi) = 
     \begin{cases}
       \text{$f(r)$, $r \in \big[0,a\big]$} \\
       \text{0, $r>a$} \\
     \end{cases}
\]

Alege'ti expresia $f(r)$ cum dori'ti 'in afar'a de o func'tie constant'a.

\end{itemize}

a) Calcula'ti vectorul induc'tiei electrice $\overline{D}$, folosind legea fluxului magnetic. \par
b) Reprezenta'ti spectrul lui $\overline{D}$. \par
c) Reprezenta'ti echivalori ale $\mid\overline{D}\mid$.

\newpage

\section{Rezolvarea circuitelor de c.a.}

\subsection{Subpunctul a)} \mbox{}

Am folosit circuitul inventat la tema 1, la care am ad'augat o bobin'a 'in serie cu rezistorul $R_1$ 'si un condensator 'in paralel cu rezistorul $R_4$. Circuitul nou format este reprezentat 'in Fig.~\ref{fig:circuit1}.

Valorile elementelor de circuit sunt:

\begin{multicols}{3}

\begin{equation}
\left\{
\begin{array}{ccl}
R_1 & = & 6\Omega \\
R_2 & = & 2\Omega\\
R_3 & = & 1\Omega \\
R_4 & = & 1\Omega
\end{array}  
\right. \nonumber % marcarea inchiderii - falsa; blocarea numerotarii
\end{equation}

\columnbreak

\begin{equation}
\left\{
\begin{array}{ccl}
J_1 & = & \SI{-6}{\ampere} \\
J_2 & = & \SI{2}{\ampere} \\
J_3 & = & \SI{3}{\ampere}
\end{array}  
\right. \nonumber % marcarea inchiderii - falsa; blocarea numerotarii
\end{equation}

\columnbreak

\begin{equation}
\left\{
\begin{array}{ccl}
E_1 & = & \SI{-2}{\volt} \\
E_2 & = & \SI{4}{\volt}
\end{array}  
\right. \nonumber % marcarea inchiderii - falsa; blocarea numerotarii
\end{equation}

\end{multicols}

% Circuitul

\begin{figure} [ht]
    \begin{center}

    \begin{circuitikz}[scale=1.2,european resistors,american inductors]
    
    \draw[black, thick]
    % coarborele
    (-2,2) to [romanianCurrentSource, l=${J_3}$] (2,2)
    (-2,-2) to [romanianCurrentSource, l=${J_2}$] (2,-2)
    (-2,-2) to [romanianCurrentSource, l=${J_1}$] (0,0)
    (-2, 2) -- (-6,2) to [R, l=${R_1}$] (-6,-2) to [L, l_=${L=600/\pi\spatiu\mathrm{mH}}$] (-2,-2)
    
    (2,2) -- (3.5,2) to [R, l=${R_4}$] (3.5,-2) -- (2,-2)
    
    (3.5,2) -- (5,2) to[C, l=${C=100/\pi\spatiu\mathrm{\mu H}}$] (5,-2) -- (3.5,-2)
    
    %nodurile
    (-2,2.3) node{(1)}
    (2,2.3) node{(2)}
    (0,0.4) node{(3)}
    (-2,-2.3) node{(4)}
    (2,-2.3) node{(5)};
    
    \draw[red, thick]
    %elementele de cicuit de pe arbore
    (-2,2) to [R, l=\color{black}${R_2}$, *-*] (-2,-2)
    (0,0) to [R, l=\color{black}${R_3}$, *-*] (2,2)
    (-2,2) to [romanianVoltageSource, l=\color{black}$E_1$] (0,0)
    (2,2) to [romanianVoltageSource, l=\color{black}$E_2$, -*] (2,-2)
    
    ;\end{circuitikz}
 \caption{Graful circuitului.}
   \label{fig:circuit1}
   \end{center}
\end{figure}

Am calculat valoarea inductivit'a'tii $L$ 'si a capacit'a'tii $C$:

\begin{equation}
\left\{
\begin{array}{ccl}
L & = & \frac{R_1 * 100}{\pi} \spatiu mH = \frac{600}{\pi} \spatiu mH \\
C & = & \frac{R_4 * 100}{\pi} \spatiu \mu F = \frac{100}{\pi} \spatiu \mu F
\end{array}  
\right. \nonumber % marcarea inchiderii - falsa; blocarea numerotarii
\end{equation} \\

Deoarece valoarea frecven'tei $f$ este egal'a cu $50 \spatiu \mathrm{Hz} \Rightarrow \omega = 2\pi f=100\pi$

\newpage

Am trecut m'arimile 'in regim sinusoidal 'si am ob'tinut urm'atoarele rezultate:

% SIC

\begin{equation}
\left\{
\begin{array}{ccl}
j_1(t) & = & -6\sqrt{2}sin(100\pi t - \frac{\pi}{2}) = -6e^{-\frac{j\pi}{2}} = -6[cos({-\frac{\pi}{2}}) + jsin({-\frac{\pi}{2}})] = 6j \\ \\
j_2(t) & = & 2\sqrt{2}sin(100\pi t + \frac{\pi}{4}) = 2e^{\frac{j\pi}{4}} = 2[cos({\frac{\pi}{4}}) + jsin({\frac{\pi}{4}})] = \sqrt{2}(1 + j) \\ \\
j_3(t) & = & 3\sqrt{2}sin(100\pi t - \frac{\pi}{4}) = 3e^{-\frac{j\pi}{4}} = 3[cos({-\frac{\pi}{4}}) + jsin({-\frac{\pi}{4}})] = \frac{3\sqrt{2}}{2}(1 - j)
\end{array}  
\right. \nonumber % marcarea inchiderii - falsa; blocarea numerotarii
\end{equation}

% SIT
\begin{equation}
\left\{
\begin{array}{ccl}
e_1(t) & = & -2\sqrt{2}sin(100\pi t + 0) = -2e^{j*0} = -2 \\ \\
e_2(t) & = & 4\sqrt{2}sin(100\pi t + \frac{\pi}{2}) = 4e^{\frac{j\pi}{2}} = 4[cos({\frac{\pi}{2}}) + jsin({\frac{\pi}{2}})] = 4j
\end{array}  
\right. \nonumber % marcarea inchiderii - falsa; blocarea numerotarii
\end{equation}

% R

\begin{equation}
\left\{
\begin{array}{ccl}
\underline{z_{R_1}} & = & R_1 = 6\Omega \\
\underline{z_{R_2}} & = & R_2 = 2\Omega\\
\underline{z_{R_3}} & = & R_3 = 1\Omega \\
\underline{z_{R_4}} & = & R_4 = 1\Omega
\end{array}  
\right. \nonumber % marcarea inchiderii - falsa; blocarea numerotarii
\end{equation}

% z_L & z_C

\begin{equation}
\left\{
\begin{array}{ccl}
\underline{z_{L}} & = & j\omega L = 60j \\
\underline{z_{C}} & = & -\frac{j}{\omega C} = -100j
\end{array}  
\right. \nonumber % marcarea inchiderii - falsa; blocarea numerotarii
\end{equation} \\

Reprezentarea 'in complex a circuitului este realizat'a 'in Fig.~\ref{fig:circuit2}.

% Circuitul

\begin{figure} [ht]
    \begin{center}

    \begin{circuitikz}[scale=1.2,european resistors,american inductors]
    
    \draw[black, thick]
    % coarborele
    (-2,2) to [romanianCurrentSource, l=${j_3(t)}$] (2,2)
    (-2,-2) to [romanianCurrentSource, l=${j_2(t)}$] (2,-2)
    (-2,-2) to [romanianCurrentSource, l=${j_1(t)}$] (0,0)
    (-2, 2) -- (-6,2) to [R, l=${z_{R_1}}$] (-6,-2) to [R, l_=${z_L}$] (-2,-2)
    
    (2,2) -- (3.5,2) to [R, l=${z_{R_4}}$] (3.5,-2) -- (2,-2)
    
    (3.5,2) -- (5,2) to[R, l=${z_C}$] (5,-2) -- (3.5,-2)
    
    %nodurile
    (-2,2.3) node{(1)}
    (2,2.3) node{(2)}
    (0,0.4) node{(3)}
    (-2,-2.3) node{(4)}
    (2,-2.3) node{(5)};
    
    \draw[red, thick]
    %elementele de cicuit de pe arbore
    (-2,2) to [R, l=\color{black}${z_{R_2}}$, *-*] (-2,-2)
    (0,0) to [R, l=\color{black}${z_{R_3}}$, *-*] (2,2)
    (-2,2) to [romanianVoltageSource, l=\color{black}$e_1(t)$] (0,0)
    (2,2) to [romanianVoltageSource, l=\color{black}$e_2(t)$, -*] (2,-2)
    
    ;\end{circuitikz}
 \caption{Circuitul 'in complex.}
   \label{fig:circuit2}
   \end{center}
\end{figure} 

\newpage

{\Large
$z_s=z_{R_1} + z_L=6+60j$ \\

$z_{p_1}=\frac{z_s*z_{R_2}}{z_s+z_{R_2}}=\frac{(6+60j)*2}{8+60j}=\frac{456+15j}{229}$ \\

$z_{p_2}=\frac{z_{R_4}*z_C}{z_{R_4}+z_C}=\frac{-100j}{1-100j} = \frac{10^4-100j}{10^4+1}$ \\
}

Circuitul devine:

% Circuitul

\begin{figure} [ht]
    \begin{center}

    \begin{circuitikz}[scale=2,european resistors,american inductors]
    
    \draw[black, thick]
    % coarborele
    (-2,2) to [romanianCurrentSource, l=${j_3(t)}$] (2,2)
    (-2,-2) to [romanianCurrentSource, l=${j_2(t)}$, *-] (2,-2)
    (-2,-2) to [romanianCurrentSource, l=${j_1(t)}$] (0,0)
    
    
    (2,2) -- (3.5,2) to [R, l=${z_{p_2}}$] (3.5,-2) -- (2,-2)
    
    %(3.5,2) -- (5,2) to[R, l=${z_C}$] (5,-2) -- (3.5,-2)
    
    %nodurile
    (-2,2.3) node{(1)}
    (2,2.3) node{(2)}
    (0,0.4) node{(3)}
    (-2,-2.3) node{(4)}
    (2,-2.3) node{(5)};
    
    \draw[red, thick]
    %elementele de cicuit de pe arbore
    (-2,2) to [R, l_=\color{black}${z_{p_1}}$, *-*] (-2,-2)
    (0,0) to [R, l=\color{black}${z_{R_3}}$, *-*] (2,2)
    (-2,2) to [romanianVoltageSource, l=\color{black}$e_1(t)$, *-] (0,0)
    (2,2) to [romanianVoltageSource, l=\color{black}$e_2(t)$, -*] (2,-2)
    
    ;\end{circuitikz}
 \caption{Circuitul 'in complex.}
   \label{fig:circuit3}
   \end{center}
\end{figure} 

\newpage

Am aplicat metoda \textit{\textbf{Kirchhoff+}}. Topologia circuitului este:

\begin{equation}
\left\{
\begin{array}{ccl} % c - centrat, l - aliniat la stanga
N & = &5 \tab noduri \\
L & = & 8 \tab laturi
\end{array}  
\right. \nonumber % marcarea inchiderii - falsa; blocarea numerotarii
\end{equation}\\

Graful intensit'a'tilor este reprezentat 'in Fig.~\ref{fig:circuit4}.

% GI

\begin{figure} [ht]
    \begin{center}

    \begin{circuitikz}[scale=1.35,european resistors,american inductors]
    
    \draw[black, thick]
    % coarborele
    (-2,2) to [short, i=$i_3$] (2,2)
    (-2,-2) to [short, i=$i_2$, *-] (2,-2)
    (-2,-2) to [short, i=$i_1$] (0,0)
    
    (2,2) -- (3.5,2) to [short, i<=$i_8$] (3.5,-2) -- (2,-2)
    
    %nodurile
    (-2,2.3) node{(1)}
    (2,2.3) node{(2)}
    (0,0.4) node{(3)}
    (-2,-2.3) node{(4)}
    (2,-2.3) node{(5)};
    
    \draw[red, thick]
    %elementele de cicuit de pe arbore
    (-2,2) to [short, i=$i_4$, color = red, *-*] (-2,-2)
    (0,0) to [short, i=$i_6$, color = red, *-*] (2,2)
    (-2,2) to [short, i=$i_5$, color = red, *-] (0,0)
    (2,-2) to [short, i=$i_7$, color = red, *-] (2, 2)
    
    ;\end{circuitikz}
 \caption{Graful de intensit'a'ti.}
   \label{fig:circuit4}
   \end{center}
\end{figure} 

Graful tensiunilor este reprezentat 'in Fig.~\ref{fig:circuit5}.

%GU

\begin{figure} [ht]
    \begin{center}

    \begin{circuitikz}[scale=1.35,european resistors,american inductors]
    
    \draw[black, thick]
    % coarborele
    (-2,2) to [short, v=$u_3$] (2,2)
    (-2,-2) to [short, v=$u_2$, *-] (2,-2)
    (-2,-2) to [short, v=$u_1$] (0,0)
    
    (2,2) -- (3.5,2) to [short, v=$u_8$] (3.5,-2) -- (2,-2)
    
    %nodurile
    (-2,2.3) node{(1)}
    (2,2.3) node{(2)}
    (0,0.4) node{(3)}
    (-2,-2.3) node{(4)}
    (2,-2.3) node{(5)};
    
    \draw[red, thick]
    %elementele de cicuit de pe arbore
    (-2,2) to [short, v<=$u_4$, color = red, *-*] (-2,-2)
    (0,0) to [short, v=$u_6$, color = red, *-*] (2,2)
    (-2,2) to [short, v=$u_5$, color = red, *-] (0,0)
    (2,2) to [short, v=$u_7$, color = red, -*] (2,-2)
    
    ;\end{circuitikz}
 \caption{Graful de tensiuni.}
   \label{fig:circuit5}
   \end{center}
\end{figure}

\newpage

Aplic\^{a}nd legea II a lui Kirchhoff pe fiecare bucl'a din sistemul fundamental, au rezultat $L-N+1 = 4$ ecua'tii din care am aflat tensiunile de pe coarde:

\begin{equation}
\left\{
\begin{array}{ccl}
i_4z_{p_1} + u_1 & = & -e_1(t) \\
i_6z_{R_3}-u_3 & = & e_1(t) \\
-i_6z_{R_3}+i_4z_{p_1}+u_2 & = & -e_1(t)-e_2(t) \\
i_8z_{p_2} & = & e_2(t)
\end{array}  
\right. \nonumber % marcarea inchiderii - falsa; blocarea numerotarii
\end{equation}

Folosind nota'tia standard pentru un sistem de ecua'tii algebrice liniare $\textbf{\textit{Ax=b}}$, am ob'tinut urm'atoarea egalitate:

\[ \left( \begin{array}{ccccc}
z_{p_2} & 0 & 0 & 0 \\
0 & 1 & 0 & 0 \\
0 & 0 & 1 & 0 \\
0 & 0 & 0 & 1
\end{array} \right)
%
\left( \begin{array}{c}
i_8 \\
u_ 1\\
u_2 \\
u_3
\end{array} \right)
=
\left( \begin{array}{c}
e_2(t) \\
-i_4z_{p_1}-e_1(t) \\
i_6z_{R_3}-i_4z_{p_1}-e_1(t)-e_2(t) \\
i_6z_{R_3}-e_1(t)
\end{array} \right),
\]
unde matricea coeficien'tilor A are dimensiunea $(4\times4)$, iar vectorul necunoscutelor 'si cel al termenilor liberi au dimensiunea $(4\times1)$.\newline

Solu'tia sistemului este:
\begin{equation}
\left\{
\begin{array}{ccl}
i_8 & = & -0.04 +  4j \\
u_1 & = & -0.33043 - 14.85631j \\
u_2 & = & -3.86596 - 18.14920j \\
u_3 & = & -1.53553 +  0.70711j
\end{array}  
\right. \nonumber % marcarea inchiderii - falsa; blocarea numerotarii
\end{equation}

\begin{center}
\includegraphics[width=15cm]{kirchhoff.png}
\end{center}

Valorile intensit'a'tilor din circuit sunt:

\begin{equation}
\left\{
\begin{array}{ccl}
\underline{i_1} & = & j_1(t) = 6j \\
\underline{i_2} & = & j_2(t) = \sqrt{2}(1+j) \\
\underline{i_3} & = & j_3(t) = \frac{3\sqrt{2}}{2}(1-j)\\
\underline{i_4} & = & i_1 + i_2 = \sqrt{2} + j(6+\sqrt{2}) \\
\underline{i_5} & = & -i_3 - i_4 = -\frac{5\sqrt{2}}{2} + j(\frac{\sqrt{2}}{2}-6) \\
\underline{i_6} & = & i_1 + i_5 = -\frac{5\sqrt{2}}{2} + j\frac{\sqrt{2}}{2} \\
\underline{i_7} & = & i_2-i_8 = \sqrt{2}(1+j) - i_8 = 1.4542 - 2.5858j\\
\underline{i_8} & = & -0.04 +  4j
\end{array}  
\right. \nonumber % marcarea inchiderii - falsa; blocarea numerotarii
\end{equation} \\

Valorile tensiunilor din circuit sunt:

\begin{equation}
\left\{
\begin{array}{ccl}
\underline{u_1} & = & -0.33043 - 14.85631j \\
\underline{u_2} & = & -3.86596 - 18.14920j \\
\underline{u_3} & = & -1.53553 +  0.70711j \\
\underline{u_4} & = & u_1 - u_5 = -2.3304 - 14.8563j \\
\underline{u_5} & = & -e_1(t) = 2 \\
\underline{u_6} & = & u_3 - u_5 = -3.53553 + 0.70711j \\
\underline{u_7} & = & -e_2(t) = -4j \\
\underline{u_8} & = & u_7 = -4j
\end{array}  
\right. \nonumber % marcarea inchiderii - falsa; blocarea numerotarii
\end{equation} 

Trec toate valorile din complex 'in scriere sinusoidal'a, cu ajutorul unei functii Octave.

\begin{center}
\includegraphics[width=15cm]{sin.png}
\end{center}


\newpage


\subsection{Subpunctul b)} \mbox{}

'In continuare, am realizat \textbf{bilan'tul de puteri} pentru a verifica faptul c'a am calculat corect elementele de circuit.

$$\underline{S_{gen}}=\sum_{k=1}^{n_{SIT}} \underline{E_k}*\underline{I_k}' + \sum_{k=1}^{n_{SIC}}\underline{U_{g_k}}*\underline{J_k}'$$
$$\underline{S_{cons}}=\sum_{k=1}^{n_{z}} \underline{z_k} * \mid \underline{I_k} \mid ^2 $$

Calculul puterilor a fost determinat cu ajutorul unei func'tii Octave.

\begin{center}
\includegraphics[width=10cm]{puteri.png}
\end{center}

\subsection{Subpunctul c)} \mbox{}

Am trasat graficul ce prezint'a evolu'tia 'in timp a parametrului $j_2(t)$.

\begin{center}
\includegraphics[width=12cm]{grafic.png}
\end{center}


\section{Rezolvarea circuitelor 'in regim tranzitoriu}

\subsection{Subpunctul a)} \mbox{}

Am folosit circuitul inventat la tema 1, la care am ad'augat o bobin'a 'in serie cu rezistorul $R_1$ 'si un condensator 'in paralel cu rezistorul $R_4$. Circuitul nou format este reprezentat 'in Fig.~\ref{fig:circuit6}.

% Circuitul

\begin{figure} [ht]
    \begin{center}

    \begin{circuitikz}[scale=1.2,european resistors,american inductors]
    
    \draw[black, thick]
    % coarborele
    (-2,2) to [romanianCurrentSource, l=${J_3}$] (2,2)
    (-2,-2) to [romanianCurrentSource, l=${J_2}$] (2,-2)
    (-2,-2) to [romanianCurrentSource, l=${J_1}$] (0,0)
    (-2, 2) -- (-6,2) to [R, l=${R_1}$] (-6,-2) to [L, l_=${L=600/\pi\spatiu\mathrm{mH}}$] (-2,-2)
    
    (2,2) -- (3.5,2) to [R, l=${R_4}$] (3.5,-2) -- (2,-2)
    
    (3.5,2) -- (5,2) to[C, l=${C=100/\pi\spatiu\mathrm{\mu H}}$] (5,-2) -- (3.5,-2)
    
    %nodurile
    (-2,2.3) node{(1)}
    (2,2.3) node{(2)}
    (0,0.4) node{(3)}
    (-2,-2.3) node{(4)}
    (2,-2.3) node{(5)};
    
    \draw[red, thick]
    %elementele de cicuit de pe arbore
    (-2,2) to [R, l=\color{black}${R_2}$, *-*] (-2,-2)
    (0,0) to [R, l=\color{black}${R_3}$, *-*] (2,2)
    (-2,2) to [romanianVoltageSource, l=\color{black}$E_1$] (0,0)
    (2,2) to [romanianVoltageSource, l=\color{black}$E_2$, -*] (2,-2)
    
    ;\end{circuitikz}
 \caption{Graful circuitului.}
   \label{fig:circuit6}
   \end{center}
\end{figure}

Am calculat valoarea inductivit'a'tii $L$ 'si a capacit'a'tii $C$:

\begin{equation}
\left\{
\begin{array}{ccl}
L & = & \frac{R_1 * 100}{\pi} \spatiu mH = \frac{600}{\pi} \spatiu mH \\
C & = & \frac{R_4 * 100}{\pi} \spatiu \mu F = \frac{100}{\pi} \spatiu \mu F
\end{array}  
\right. \nonumber % marcarea inchiderii - falsa; blocarea numerotarii
\end{equation} \\

Am considerat c'a la momentul $t=0$ apare un defect 'in circuit 'si se rupe latura ce con'tine sursa ideal'a de tensiune $E_2$. \\

La $t=0_-$, circuitul este 'in curent continuu. Bobina se comport'a ca un conductor perfect ($R\rightarrow0$), 'in timp ce condensatorul ca un izolator perfect ($R\rightarrow\infty$). Latura nu este 'inc'a rupt'a. Circuitul la $t=0_-$ este ilustrat 'in Fig.~\ref{fig:circuit7}.

Circuitul este identic cu cel de la tema 1, de unde rezult'a:

\begin{equation}
\left\{
\begin{array}{ccl}
u_{C_{(0_-)}} & = & \SI{4}{\volt} \\
i_{L_{(0_-)}} & = & \SI{1}{\ampere}
\end{array}  
\right. \nonumber % marcarea inchiderii - falsa; blocarea numerotarii
\end{equation}

% Circuitul

\begin{figure} [ht]
    \begin{center}

    \begin{circuitikz}[scale=1.2,european resistors,american inductors]
    
    \draw[black, thick]
    % coarborele
    (-2,2) to [romanianCurrentSource, l=${J_3}$] (2,2)
    (-2,-2) to [romanianCurrentSource, l=${J_2}$] (2,-2)
    (-2,-2) to [romanianCurrentSource, l=${J_1}$] (0,0)
    (-2, 2) -- (-6,2) to [R, l=${R_1}$] (-6,-2) to [short, i<=$i_{L_{(0_-)}}$] (-2,-2)
    
    (2,2) -- (3.5,2) to [R, l=${R_4}$] (3.5,-2) -- (2,-2)
    
    (3.5,2) -- (5,2) -- (5, 0.5)
    (5, -0.5) -- (5,-2) -- (3.5,-2)
    
    (5, 0.5) to [open, v^<=$u_{C_{(0_-)}}$,*-*] (5, -0.5)
    
    %nodurile
    (-2,2.3) node{(1)}
    (2,2.3) node{(2)}
    (0,0.4) node{(3)}
    (-2,-2.3) node{(4)}
    (2,-2.3) node{(5)};
    
    \draw[red, thick]
    %elementele de cicuit de pe arbore
    (-2,2) to [R, l=\color{black}${R_2}$, *-*] (-2,-2)
    (0,0) to [R, l=\color{black}${R_3}$, *-*] (2,2)
    (-2,2) to [romanianVoltageSource, l=\color{black}$E_1$] (0,0)
    (2,2) to [romanianVoltageSource, l=\color{black}$E_2$, -*] (2,-2)
    
    ;\end{circuitikz}
 \caption{Graful circuitului la $t=0_-$}
   \label{fig:circuit7}
   \end{center}
\end{figure}

La $t \rightarrow \infty$, circuitul este 'in curent continuu. Precum la $t=0_-$, bobina se comport'a ca un conductor perfect ($R\rightarrow0$), iar condensatorul ca un izolator perfect ($R\rightarrow\infty$). Latura ce con'tine sursa $E_2$ este rupt'a. Graful circuitului la $t \rightarrow \infty$ este reprezentat 'in Fig.~\ref{fig:circuit9}.

% Circuitul

\begin{figure} [ht]
    \begin{center}

    \begin{circuitikz}[scale=1.2,european resistors,american inductors]
    
    \draw[black, thick]
    % coarborele
    (-2,2) to [romanianCurrentSource, l=${J_3}$] (2,2)
    (-2,-2) to [romanianCurrentSource, l=${J_2}$] (2,-2)
    (-2,-2) to [romanianCurrentSource, l=${J_1}$] (0,0)
    (-2, 2) -- (-6,2) to [R, l=${R_1}$] (-6,-2) to [short, i<=$i_{L_{\infty}}$] (-2,-2)
    
    (2,2) -- (3.5,2) -- (3.5, 0.5) to [open, v^<=$u_{C_{\infty}}$,*-*] (3.5, -0.5) -- (3.5,-2) -- (2,-2)
    
    %nodurile
    (-2,2.3) node{(1)}
    (2,2.3) node{(2)}
    (0,0.4) node{(3)}
    (-2,-2.3) node{(4)}
    (2,-2.3) node{(5)};
    
    \draw[red, thick]
    %elementele de cicuit de pe arbore
    (-2,2) to [R, l=\color{black}${R_2}$, *-*] (-2,-2)
    (0,0) to [R, l=\color{black}${R_3}$, *-*] (2,2)
    (-2,2) to [romanianVoltageSource, l=\color{black}$E_1$] (0,0)
    %(2,2) to [romanianVoltageSource, l=\color{black}$E_2$, -*] (2,-2)
    (2,2) to [R, l=\color{black}${R_4}$] (2,-2)
    
    ;\end{circuitikz}
 \caption{Graful circuitului la $t \rightarrow \infty$}
   \label{fig:circuit9}
   \end{center}
\end{figure}

\newpage

Pentru a calcula valorile intensit'a'tilor 'si tensiunilor la $t \rightarrow \infty$ am folosit simulatorul Spice.

\includegraphics[width=2.5cm]{netlist_t_inf.PNG} \TAB
\includegraphics[width=7.5cm]{result_netlist_t_inf.PNG} \\

Graful de intensit'a'ti 'si de tensiuni sunt reprezentate 'in Fig.~\ref{fig:circuit10}, resprectiv 'in Fig.~\ref{fig:circuit11}.

% Circuitul

\begin{figure} [ht]
    \begin{center}

    \begin{circuitikz}[scale=1,european resistors,american inductors]
    
    \draw[black, thick]
    % coarborele
    (-2,2) to [short, i>=\SI{3}{\ampere}] (2,2)
    (-2,-2) to [short, i>=\SI{2}{\ampere}] (2,-2)
    (-2,-2) to [short, i<=\SI{6}{\ampere}] (0,0)
    (-2, 2) -- (-6,2) -- (-6,-2) to [short, i<=\SI{1}{\ampere}] (-2,-2)
    
    %(2,2) -- (3.5,2) -- (3.5, 0.5) to [open, v^<=$u_{C_{\infty}}$,*-*] (3.5, -0.5) -- (3.5,-2) -- (2,-2)
    
    %nodurile
    (-2,2.3) node{(1)}
    (2,2.3) node{(2)}
    (0,0.4) node{(3)}
    (-2,-2.3) node{(4)}
    (2,-2.3) node{(0)};
    
    \draw[red, thick]
    %elementele de cicuit de pe arbore
    (-2,2) to [short, i<=\color{black}\SI{3}{\ampere}, *-*] (-2,-2)
    (0,0) to [short, i<=\color{black}\SI{5}{\ampere}, *-*] (2,2)
    (-2,2) to [short, i>=\color{black}\SI{1}{\ampere}] (0,0)
    %(2,2) to [romanianVoltageSource, l=\color{black}$E_2$, -*] (2,-2)
    (2,2) to [short, i<=\color{black}\SI{2}{\ampere}] (2,-2)
    
    ;\end{circuitikz}
 \caption{Graful intensit'a'tilor la $t \rightarrow \infty$}
   \label{fig:circuit10}
   \end{center}
\end{figure}

% Circuitul

\begin{figure} [ht]
    \begin{center}

    \begin{circuitikz}[scale=1,european resistors,american inductors]
    
    \draw[black, thick]
    % coarborele
    (-2,2) to [short, v=\SI{-3}{\volt}, *-*] (2,2)
    (-2,-2) to [short, v^<=\SI{-1}{\volt}, *-*] (2,-2)
    (-2,-2) to [short, v<=\SI{-8}{\volt}, *-*] (0,0)
    (-2, 2) -- (-6,2) -- (-6,-2) -- (-2,-2)
    
    (-2,2) to [short, v<=\color{black}\SI{6}{\volt}, color = black, *-*] (-2,-2)
    
    %nodurile
    (-2,2.3) node{(1)}
    (2,2.3) node{(2)}
    (0,0.4) node{(3)}
    (-2,-2.3) node{(4)}
    (2,-2.3) node{(0)};
    
    \draw[red, thick]
    %elementele de cicuit de pe arbore
    (-2,2) to [short, v^<=\SI{6}{\volt}, color = red, *-*] (-2,-2)
    (0,0) to [short, v^<=\SI{5}{\volt}, color = red, *-*] (2,2)
    (-2,2) to [short, v^>=\SI{2}{\volt}, color = red, *-*] (0,0)
    %(2,2) to [romanianVoltageSource, l=\color{black}$E_2$, -*] (2,-2)
    (2,2) to [short, v>=\SI{-2}{\volt}, color = red, *-*] (2,-2)
    
    ;\end{circuitikz}
 \caption{Graful tensiunilor la $t \rightarrow \infty$}
   \label{fig:circuit11}
   \end{center}
\end{figure}

Rezult'a urm'atoarele valori pentru parametri c'auta'ti:

\begin{equation}
\left\{
\begin{array}{ccl}
u_{C_{\infty}} & = & \SI{2}{\volt} \\
i_{L_{\infty}} & = & \SI{1}{\ampere}
\end{array}  
\right. \nonumber % marcarea inchiderii - falsa; blocarea numerotarii
\end{equation} \\

La $t=0_+$, latura este rupt'a. Reprezentarea circuitului 'in opera'tional este ilustrat'a 'in Fig.~\ref{fig:circuit8}.

% Circuitul

\begin{figure} [ht]
    \begin{center}

    \begin{circuitikz}[scale=1.2,european resistors,american inductors]
    
    \draw[black, thick]
    % coarborele
    (-2,2) to [romanianCurrentSource, l=${J_3}$] (2,2)
    (-2,-2) to [romanianCurrentSource, l=${J_2}$] (2,-2)
    (-2,-2) to [romanianCurrentSource, l=${J_1}$] (0,0)
    (-2, 2) -- (-6,2) to [R, l=${R_1 + z_L(s)}$] (-6,-2)
    
    (-2,-2) to [romanianVoltageSource, l=\color{black}$E_L(s)$] (-6,-2)
    
    (2,2) -- (6,2) to [romanianVoltageSource, l=\color{black}$E_C(s)$] (6,-2) to [R, l=${z_C(s)}$] (2,-2)
    
    %nodurile
    (-2,2.3) node{(1)}
    (2,2.3) node{(2)}
    (0,0.4) node{(3)}
    (-2,-2.3) node{(4)}
    (2,-2.3) node{(5)};
    
    \draw[red, thick]
    %elementele de cicuit de pe arbore
    (-2,2) to [R, l=\color{black}${R_2}$, *-*] (-2,-2)
    (0,0) to [R, l=\color{black}${R_3}$, *-*] (2,2)
    (-2,2) to [romanianVoltageSource, l=\color{black}$E_1$] (0,0)
    %(2,2) to [romanianVoltageSource, l=\color{black}$E_2$, -*] (2,-2)
    (2,2) to [R, l=\color{black}${R_4}$] (2,-2)
    
    ;\end{circuitikz}
 \caption{Reprezentarea circuitului 'in opera'tional.}
   \label{fig:circuit8}
   \end{center}
\end{figure}

Parametri bobin'a:

\begin{equation}
\left\{
\begin{array}{ccl}
z_L(s) & = & s*L = s * \frac{600 * 10^{-3}}{\pi}= \frac{0.6s}{\pi} \\
E_L(s) & = & L * i_{L_{(0_-)}} = \frac{600 * 10^{-3}}{\pi} * 1 = \frac{0.6}{\pi}
\end{array}  
\right. \nonumber % marcarea inchiderii - falsa; blocarea numerotarii
\end{equation} \\

Parametri condensator:

\begin{equation}
\left\{
\begin{array}{ccl}
z_C(s) & = & \frac{1}{sC} = \frac{1}{s * \frac{100 * 10^{-6}}{\pi}}= \frac{10^{4}\pi}{s} \\
E_C(s) & = & \frac{u_{C_{(0_-)}}}{s} = \frac{4}{s}
\end{array}  
\right. \nonumber % marcarea inchiderii - falsa; blocarea numerotarii
\end{equation} \\

Am aplicat metoda Kirchhoff pentru a determina intensit'a'tile de pe laturile care nu sunt de tip SIC 'si tensiunile de pe laturile de tip SIC.

Folosind nota'tia standard pentru un sistem de ecua'tii algebrice liniare $\textbf{\textit{Ax=b}}$, am ob'tinut urm'atoarea egalitate:

\[ \left( \begin{array}{ccccccccc}
1 & 1 & 0 & 0 & 0 & -1 & 0 & 0 & 0 \\
0 & 0 & 1 & 1 & 1 & 0 & 0 & 0 & 0 \\
0 & 0 & 0 & 0 & -1 & 1 & 0 & 0 & 0 \\
1 & 1 & 0 & 0 & 0 & 0 & 0 & 0 & 0 \\
4 + \frac{3s}{5\pi} & -2 & 0 & 0 & 0 & 0 & 0 & 0 & 0 \\
0 & 2 & 0 & 0 & 0 & 0 & 1 & 0 & 0 \\
0 & 0 & 0 & 0 & 1 & 0 & 0 & 0 & 1 \\
0 & 0 & 1 & 0 & -1 & 0 & 1 & 1 & 0 \\
0 & 0 & -1 & \frac{10000\pi}{s} & 0 & 0 & 0 & 0 & 0 
\end{array} \right)
%
\left( \begin{array}{c}
I_1(s) \\
I_2(s) \\
I_6(s) \\
I_7(s) \\
I_8(s) \\
I_9(s) \\
U_1(s) \\
U_2(s) \\
U_3(s)
\end{array} \right)
=
\left( \begin{array}{c}
\frac{3}{s} \\
\frac{-3}{s} \\
\frac{6}{s} \\
\frac{4}{s} \\
\frac{0.6}{\pi} \\
\frac{-2}{s} \\
\frac{-2}{s} \\
0 \\
-\frac{4}{s} \\
\end{array} \right),
\]
unde matricea coeficien'tilor A are dimensiunea $(9\times9)$, iar vectorul necunoscutelor 'si cel al termenilor liberi au dimensiunea $(9\times1)$.\newline

\begin{center}
\includegraphics[width=6cm]{syms1.png}
\end{center}

\begin{center}
\includegraphics[width=15cm]{syms2.png}
\end{center}

\subsection{Subpunctul b)} \mbox{}

Cu ajutorul unui cod Octave, am verificat daca este respectat'a teorema valorilor ini'tiale 'si finale. Valorile ini'tiale se determin'a la $t \rightarrow 0$, iar cele finale la $t \rightarrow \infty$. \\

\includegraphics[width=4cm]{val_init_val_final.png}

Din poza de mai sus se poate observa c'a sistemul urm'ator este verificat:

\begin{equation}
\left\{
\begin{array}{ccl}
i_{L_{(0_-)}} & = & \SI{1}{\ampere} \\
u_{C_{(0_-)}} & = & \SI{4}{\volt} \\
i_{L_{\infty}} & = & \SI{1}{\ampere} \\
u_{C_{\infty}} & = & \SI{2}{\volt}
\end{array}  
\right. \nonumber % marcarea inchiderii - falsa; blocarea numerotarii
\end{equation} \\

\subsection{Subpunctul c)} \mbox{}

Am calculat expresia instantanee a variabilei $I_L(s)$ 'si am reprezentat-o grafic.

\begin{center}
\includegraphics[width=9cm]{expresie.png}
\end{center}

\begin{center}
\includegraphics[width=15cm]{grafic_laplace.png}
\end{center}

\section{Calculul 'si reprezentarea unui c\^amp electric} \mbox{}

\subsection{Subpunctul a)} \mbox{}

Am ales func'tia $f(r)=r^3+7$. Distribu'tia de sarcin'a devine:

\[   
\rho(r, \theta, \phi) = 
     \begin{cases}
       \text{$r^3+7$, $r \in \big[0,a\big]$} \\
       \text{0, $r>a$} \\
     \end{cases}
\]\\

Calculez vectorul induc'tiei electrice $\overline{D}$ folosind legea fluxului electric. \par
Am considerat o sfer'a $\Sigma$ de raz'a $a$. 
\newpage
{\large
$\psi_{\Sigma} = q_{D_{\Sigma}}$ \\

$\int_{\Sigma}\overline{D}\cdot\overline{dA} = \int_{D_{\Sigma}} \rho dV$ \\

$\int_{\Sigma}\overline{D}\cdot\overline{dA} = D \int_{\Sigma}dA = D \cdot 4\pi r^2$\\

Tratez dou'a cazuri:

1) $r \in \big[0,a\big]$

$q_{D_{\Sigma}} = \int_{D_{\Sigma}} \rho dV = \int_{0}^{r}(t^3+7)4\pi t^2 dt = 4\pi \int_{0}^{r}(t^5+7t^2)dt = 4\pi (\frac{r^6}{6} + \frac{7r^3}{3})$ \\

2) $r > a$

$q_{D_{\Sigma}} = \int_{D_{\Sigma}} \rho dV = \int_{0}^{a}(r^3+7)4\pi r^2 dr = 4\pi \int_{0}^{a}(r^5+7r^2)dr = 4\pi (\frac{a^6}{6} + \frac{7a^3}{3})$ \\

$D(r) = \frac{q_{D_{\Sigma}}}{4\pi r^2}$

\[   
D(r) = 
     \begin{cases}
       \text{$\frac{r^4}{6} + \frac{7r}{3}$, $r \in \big[0,a\big]$} \\
       \text{$\frac{1}{r^2}(\frac{a^6}{6} + \frac{7a^3}{3})$, $r>a$} \\
     \end{cases}
\]}

\subsection{Subpunctul b)} \mbox{}
Am reprezentat spectrul lui $\overline{D}$.

\begin{center}
\includegraphics[width=10cm]{spectru.png}
\end{center}

\subsection{Subpunctul c)} \mbox{}
Am reprezentat echivalori ale lui $\mid\overline{D}\mid$.

\begin{center}
\includegraphics[width=12cm]{echivalori.png}
\end{center}

Codul aferent exerci'tiului 3 poate fi observat mai jos. \\

\includegraphics[width=5cm]{camp.png} \TAB
\includegraphics[width=7.5cm]{functie.png} \\

\newpage

\section{Redactarea \^{i}n \LaTeX} \mbox{}

Tema a fost realizat'a \^{i}n \LaTeX, un editor text de \^{i}nalt'a performan't'a. Mai jos, se poate observa un fragment de cod utilizat pentru generarea acestui fi'sier PDF. \\

\begin{center}
\includegraphics[width=15cm]{latex.png}
\end{center}

\newpage

%Bibliografie

\begin{thebibliography}{9}
\bibitem{latexcompanion} 
Daniel Ioan,
\textit{Circuite electrice rezistive - breviare teoretice 'si probleme}, 
\\\texttt{http://www.lmn.pub.ro/daniel/culegere.pdf}, 2000.
 
\bibitem{einstein} 
G. Ciuprina, A. Gheorghe, M. Popescu, D. Niculae, A.S. Lup, R. B'arbulescu, D. Ioan,
\textit{Modelarea 'si simularea circuitelor electrice. 'Indrumar de laborator}, \\
\texttt{http://cs.curs.pub.ro/2017/course/view.php?id=50}
 
\bibitem{knuthwebsite} 
Gabriela Ciuprina,
\textit{Template pentru redactarea rapoartelor \^{i}n LaTeX (v4)}, \\
\texttt{http://cs.curs.pub.ro/2017/course/view.php?id=50}
\end{thebibliography}

\end{document}

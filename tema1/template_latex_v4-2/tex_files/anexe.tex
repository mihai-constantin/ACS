\section{Preambul folosit pentru generarea acestui document}
\label{sec:preambul}

Preambulul folosit pentru generarea acestui document este urm'atorul:

\begin{small}
\begin{verbatim}
\documentclass[12pt,twoside]{article}  
\usepackage{amsmath,epsfig,pifont,calc,pifont,pstricks}
\usepackage{media9}
\usepackage{animate} % daca includeti animatii
\usepackage{graphicx}	
\usepackage[romanian]{babel}
\usepackage[unicode]{hyperref} 
\usepackage{rom} 		 % pentru a scrie cu diacritice in limba romana	

%%% setari ale paginii
\setlength{\parindent}{3ex}
%dimensiunea textului pe pagina
\setlength{\voffset}{-2cm}
\setlength{\textheight}{23cm}  
\setlength{\textwidth}{16cm}
\setlength{\topmargin}{0cm}
\setlength{\headsep}{1cm}
\renewcommand{\baselinestretch}{1.2}
\newcommand{\myindent}{\hspace*{3ex}}
%\renewcommand{\baselinestretch}{1}

%margini
\setlength{\oddsidemargin}{0.5cm}
\setlength{\evensidemargin}{-0.3cm}
%\raggedright
\raggedbottom

%% - macro-uri definite de autor
\newcommand{\D}{\mathrm{d}}	% va fi folosita in mediul matematic, 
                            % pentru diferentiala d
\newcommand{\I}{\mathrm{i}}	% va fi folosita in mediul matematic, 
                            % pentru unitatea imaginara
\newcommand{\eul}{\mathrm{e}}	% numarul lui Euler
\newcommand{\vect}[1]{\mathbf{#1}}	% comanda cu un argument, 
                            % pentru scrierea vectorilor cu lidere aldine, drepte
                            % comanda \vec a LaTeX pune sageti deasupra.
																										
\newcommand{\reale}{\mbox{${\scriptstyle \rm I\!R}$}}  % multimea numerelor reale
\newcommand{\complexe}{\mbox{${\scriptstyle \rm I\!\!\!\!C}$}}
\newcommand{\rationale}{\mbox{${\scriptstyle \rm I\!\!\!\!Q}$}}
\newcommand{\naturale}{\mbox{${\scriptstyle \rm I\!\!N}$}}
\newcommand{\intregi}{\mbox{${\scriptstyle \rm Z\!\!\!Z}$}}
%% end preambul
\end{verbatim}
\end{small}

\section{Cod Matlab}
\label{sec:codMatlab}

Acesta este un cod matlab. Pute'ti descoperi 'si singuri ce face.

\lstinputlisting{tex_files/cod/codMatlab.m}

\subsubsection{Circuite electrice}

Pute'ti desena circuite cu ajutorul pachetul circuitiks,
disponibil la\\ \href{https://www.ctan.org/pkg/circuitikz}{https://www.ctan.org/pkg/circuitikz}

'In Fig.\ref{fig:circuit1} ave'ti un exemplu de astfel de schem'a, 'in care sunt folosite simboluri IEEE.
\begin{figure}[b]
\begin{center}
 \begin{circuitikz}[scale=1.4,american]\draw
 (0,0) to[C, l=10<\micro\farad>] (0,2) -- (0,3)
 to[R, l=2.2<\kilo\ohm>] (4,3) -- (4,2);
 \draw (4,0) to[L, l_=12<\milli\henry>, i<_=$i_1$] (4,2);
 \draw (4,0) -- (0,0)
 (4,2) { to[D*, *-*, color=red] (2,0) }
 (0,2) to[R, l=1<\kilo\ohm>, *-] (2,2)
 to[V, i_=$i_2$,v^=$e(t)$] (4,2)
 (2,0) to[I, l=$j(t)$, -*] (2,2)
  ;\end{circuitikz}
  \caption{Circuit realizat cu circuitiks, simboluri "americane".}
  \label{fig:circuit1}
  \end{center}
  \end{figure}

Dac'a dori'ti s'a folosi'ti simboluri ca 'in c'ar'tile clasice scrise 'in limba
rom\^an'a, atunci pute'ti adauga pachetului circuitiks simbolurile create de Adrian Pop,
disponibile la
\href{https://github.com/PopAdi/circuitikz-romanian-symbols}{https://github.com/PopAdi/circuitikz-romanian-symbols}.

Acela'si circuit, desenat cu astfel de simboluri, este cel din
Fig.\ref{fig:circuit2}.

\begin{figure}
\begin{center}
\begin{circuitikz}[scale=1.4,european resistors,american inductors]\draw
(0,0) to[C, l=10<\micro\farad>] (0,2) -- (0,3)
to[R, l=2.2<\kilo\ohm>] (4,3) -- (4,2);
\draw (4,0) to[L, l_=12<\milli\henry>, i<_=$i_1$,v^=$u_1$] (4,2);
\draw (4,0) -- (0,0)
(4,2) { to[D*, *-*, color=red] (2,0) }
(0,2) to[R, l=1<\kilo\ohm>, *-] (2,2)
to[romanianVoltageSource, i_=$i_2$,v^=$e(t)$] (4,2)
(2,0) to[romanianCurrentSource, l=$j(t)$, -*] (2,2)
 ;\end{circuitikz}
 \caption{Circuit realizat cu circuitiks, simboluri utilizate 'in c'ar'ti clasice 'in limba rom\^an'a.}
   \label{fig:circuit2}
   \end{center}
   \end{figure}

 